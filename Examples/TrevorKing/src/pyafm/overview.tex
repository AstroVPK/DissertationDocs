Velocity clamp experiments have been carried out since the initial
work by \citet{rief97a}, so I was somewhat surprised that there
weren't already community-driven packages for carrying out and
analyzing these
experiments\citep{claerbout92,buckheit95,schwab00,vandewalle09}.  When
I joined Prof.~Yang's lab, we were using experiment control software
written in \citetalias{labview} and analysis software written in
\citetalias{wavemetrics-igor}, both developed in-house.  The existing
software was not designed to control sample temperature or for easy
extension, so I proceeded to write my own control and analysis stack
to add these capabilities.

For those of you thinking, ``Why is he calling this thing a stack?'',
software is rarely developed as a single monolithic program.  Instead,
developers write software as a series of modular components, with each
layer in the stack using lower level features from the layers below it
to supply higher level features to the layers above it.  New
high-level programs will contain logic for the new idea (perform
velocity-clamp unfolding experiments) and leverage pre-existing
packages for all the old ideas that you need to get the job done (open
a file, Fourier transform an array, \ldots).  A well structured suite
of software breaks up the task at hand into many sub-components, with
a distinct package handling each component.

\citet{whitehead11} introduces his claim about civilization and
subconscious operations to motivate the utility of symbolism in
subconcious reasoning.  By encapsulating already established ideas in
a compact form, we can focus on the crux of an issue without being
distracted by the peripheral boilerplate.

In this chapter, I will discuss the earlier frameworks and abortive
attempts that lead me towards my current architecture
(\cref{sec:aio-frameworks,sec:pyafm:stack}).  I will also discuss some
auxiliary packages I developed to support the main stack
(\cref{sec:pyafm:auxiliary}).  I'll wrap up by comparing my stack with
Prof.~Yang's earlier framework and summarizing lessons I've learned
along the way (\cref{sec:pyafm:discussion,sec:pyafm:conclusions}).
