\section{Hardware}
\label{sec:future:hardware}

Very few approach--bind--retract cycles actually pick up a protein and
produce a clean unfolding curve.  This makes it hard to gather
sufficient unfolding statistics unless you can run experiments
continuously over a long time span.  While our modified
MultiMode\citep{multimode} gets the job done, some hardware upgrades
would allow futher automation, increasing throughput through longer
run times.

MultiModes measure the position of the cantilever by monitoring the
reflected laser with a four-segment photodiode.  While the vertical
and horizontal signals are accessible in the cable connecting the
MultiMode to its controlling NanoScope, the total photodiode signal is
not.  The loss of laser signal---which can occur when bubbles in the
fluid cell obstruct the laser---results in low voltage deflection that
is independent of piezo position.  This flat-line deflection also
occurs when mechanical drift moves the surface out of range for the
piezo positioner.  In the drift case, we would like to use the stepper
motor to reduce the tip--surface separation.  In the loss-of-signal
case, we would like to \emph{increase} the tip--surface separation to
avoid accidentally crashing the tip into a surface we can no longer
detect.  By exposing the total photodiode signal to the control
software, we could unambiguously distinguish these two cases.  This
would allow for longer runs, aggressively using the stepper motor to
mitigate mechanical drift.

We could also reduce deflection signal noise---which is especially
important for accurate cantilever calibration
(\cref{sec:calibcant})---by automating photodiode positioning.  The
four-segment photodiode has the least signal noise when the deflected
laser lands near the point between all four sections.  However,
mechanical drift in microscope alignment causes the spot location to
vary with time.  We currently use manual thumbscrews to re-zero the
photodiode as needed, but unmonitored overnight runs would require
computer-controlled positioning.  Similar automatic positioning would
be useful for automatically aligning the incoming laser with the
cantilever.  While laser--cantilever alignment seems to be less
sensitive than cantilever--photodiode alignment, automatic laser
alignment would also open the door to automatic piezo calibration
through measurements of laser interference patterns\citep{jaschke95}.

Finally, our current hardware does not address potential piezo
hysteresis, nonlinearity, or drift.  Newer piezos often use capacitive
feedback, adjusting the driving voltage as needed to maintain the
target extension.  Besides making existing distance measurements more
accurate, the increased stability opens the door to slower pulling
speeds needed to monitor proteins with less stable folded positions.
