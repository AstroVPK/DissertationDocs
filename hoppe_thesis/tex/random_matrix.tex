\section{Random Matrices}

Random matrices were brought to the forefront of physics due to the realization of Wigner that the eigenvalues of the Hamilton of complex atomic spectra could be modeled as the eigenvalues of random matrices. Typical  of random matrices are the so-called level repulsion, that the eigenvalues are more evenly spread across their domain then a random selection would have.

The field is broken up into two distinct categories, matrices that are Hermitian and those that are not. Since the eigenvalues of a Hermitian matrix are all real this allows a simple expansion to determine the spectrum when $N$ is large.

Defining a few simple operators:

\begin{equation}
  \partial    = \pfrac{}{z}   = \frac{1}{z} \paren{ \pfrac{}{x} - i \pfrac{}{y} }
  \partial^* = \pfrac{}{z^*} = \frac{1}{z} \paren{ \pfrac{}{x} + i \pfrac{}{y} }
\end{equation}

Then the density of eigenvalues can be given by:

\begin{equation}
  \rho(x,y) = \frac{1}{\pi} \partial \partial ^* 
  \avg{ \frac{1}{N} Tr \log (z-\phi)(z^* -\phi^ \dagger) }
\end{equation}



\subsection{Hermitian Matrices}

\subsection{Non-Hermitian - Ginibre Ensemble}
Consider a matrix $\phi$ drawn from a probability distribution:

\begin{equation}
  \frac{1}{z} \exp{ -N Tr \phi^ \dagger \phi  }
\end{equation}

\subsection{Non-Hermitian - Rate equations}



\subsection{Transient States}
\subsection{Kinetic Pathways}
