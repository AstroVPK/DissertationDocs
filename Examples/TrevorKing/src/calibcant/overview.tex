The most common method for calibrating cantilevers for atomic force
microscopes is via thermal vibration\citep{florin95}.  In this
chapter, I'll derive the theory behind this procedure and introduce
my \calibcant\ package for performing this calibration automatically.

We know the energy of the cantilever's thermal vibration from the
equipartion theorem (\cref{eq:equipart,sec:cantilever-calib:intro}).
Solving the equipartition theorem for the spring contant $\kappa$
yields
\begin{equation}
  \kappa = \frac{k_BT}{\avg{x^2}} \;, \label{eq:equipart_k}
\end{equation}
so we need to measure (or estimate) the temperature $T$ and variance
of the cantilever position $\avg{x^2}$ in order to estimate $\kappa$.

We don't measure $x$ directly, though.  We reflect a laser off the
back of the cantilever and measure the position of the deflected beam
with a photodiode (\cref{fig:afm-schematic}).  In order to convert the
photodiode signal $V_p$\index{$V_p$} to a tip displacement $x$, we
scale $V_p$ by a linear photodiode sensitivity
$\sigma_p$\index{$\sigma_p$}.
\begin{equation}
  x(t) = \frac{V_p(t)}{\sigma_p} \;. \label{eq:x-from-Vp}
\end{equation}
We measure $\sigma_p$ by pushing the tip against the substrate surface
and measuring the slope (deflection volts per piezo meter) of the
resulting contact-deflection trace (\cref{sec:calibcant:bump}).  By
keeping $V_p$ and $\sigma_p$ separate in our calculation of $\kappa$,
we can gauge the relative importance errors in each parameter and
calculate the uncertainty in our estimated $\kappa$
(\cref{sec:calibcant:discussion:errors}).
%
\nomenclature[sr ]{$V_p$}{The vertical photodiode deflection voltage
  (\cref{fig:afm-schematic,eq:x-from-Vp}).}
\nomenclature[sg s_p ]{$\sigma_p$}{The linear photodiode sensitivity to
  cantilever displacement (\cref{fig:afm-schematic,eq:x-from-Vp}).}

In order to filter out noise in the measured value of $\avg{V_p^2}$ we
fit the measured cantilever deflection to the expected theoretical
power spectral density ($\PSD_f$\index{PSD@\PSD!in frequency space})
of a damped harmonic oscillator exposed to thermal noise
\begin{equation}
  \PSD_f(V_p, f) = \frac{G_{1f}}{(f_0^2-f^2)^2 + \beta_f^2 f^2} \;.
  \label{eq:psd-Vp}
\end{equation}
In terms of the fit parameters $G_{1f}$\index{$G_{1f}$},
$f_0$\index{$f_0$}, and $\beta_f$\index{$\beta_f$}, the expectation
value for $V_p^2$ is given by
\begin{equation}
  \avg{V_p(t)^2} = \frac{\pi G_{1f}}{2\beta_f f_0^2} \;.
  \label{eq:avg-Vp-Gone-f}
\end{equation}
%
\nomenclature[o PSDf ]{$\PSD_f$}{Power spectral density in
  frequency space
  \begin{equation}
    \PSD_f(g, f) \equiv \normLimT 2 \magSq{ \Fourf{g(t)}(f) } \;.
  \end{equation}}
\nomenclature[sr ]{$f$}{Frequency (hertz).}
\nomenclature[sr ]{$f_0$}{Resonant frequency (hertz).}
\nomenclature[sg p ]{$\pi$}{Archmides' constant, $\pi=3.14159\ldots$.
  The ratio of a circle's circumference to its diameter.}

Combining \cref{eq:equipart_k,eq:x-from-Vp,eq:avg-Vp-Gone-f}, we
have
\begin{align}
  \kappa &= \frac{\sigma_p^2 k_BT}{\avg{V_p(t)^2}}
    = \frac{2 \beta_f f_0^2 \sigma_p^2 k_BT}{\pi G_{1f}} \;.
  \label{eq:kappa}
\end{align}
A calibration run consists of bumping the surface with the cantilever
tip to measure $\sigma_p$ (\cref{sec:calibcant:bump}), measuring the
buffer temperature $T$ with a thermocouple
(\cref{sec:calibcant:temperature}), and measuring thermal vibration
when the tip is far from the surface to extract the fit parameters
$G_{1f}$, $f_0$, and $\beta_f$ (\cref{sec:calibcant:vibration}).

Although this theory should be well established
(\cref{sec:calibcant:survey}), there is continued confusion about the
details of the fitting (\cref{sec:calibcant:lorentzian}).  To avoid
further ambiguity, I'll derive the power spectral density mentioned
above (\cref{eq:psd-Vp,eq:avg-Vp-Gone-f})
in \cref{sec:calibcant:theory}.  In \cref{sec:calibcant:procedure},
I'll introduce my \calibcant\ package for automatically calibrating
cantilevers.  I'll clear up a few remaining points
in \cref{sec:calibcant:discussion} before wrapping up
with \cref{sec:calibcant:conclusions}.

\section{Related work}
\label{sec:calibcant:survey}

In reality, the cantilever motion is more complicated than a pure
simple harmonic oscillator.  Various corrections taking into acount
higher order vibrational modes\citep{butt95,stark01} and cantilever
tilt\citep{hutter05} have been proposed and
reviewed\citep{florin95,levy02,ohler07}, but we will focus here on the
derivation of noise in damped simple harmonic oscillators that
underlies all frequency-space methods for improving the basic
$\kappa\avg{x^2} = k_BT$ method.

\citet{roters96} derive the \PSD\ with a similar Fourier transform,
but they use the fluctuation--dissipation theorem to extract the \PSD\
from the susceptibility (see
their \fref{equation}{4}).  \citet{benedetti12} has independently
developed a Parseval's approach similar to mine (in
his \fref{section}{8.2.1}), although he glosses over some of the
integrals.  \citet{berg-sorensen04} has an extensive treatment of the
extremely overdamped case and laser tweezer calibration, which they
revisit a year later during a discussion of noise
color\citep{berg-sorensen05}.  \citet{gittes98} derive some related
results in the extremely overdamped case, such the fact that the
signal to thermal noise ratio is independent of trap stiffness
$\kappa$.  Despite this earlier work, I think it is worth explicitly
deriving the \PSD\ of a damped harmonic oscillator here, as I have
been unable to find a reference that I feel treats the problem with
sufficient rigor.  An explicit derivation may also help clear up the
confusion about the proper \PSD\ form discussed in the next section.

\section{Fitting with a Lorentzian}
\label{sec:calibcant:lorentzian}

It is popular to refer to the thermal power spectral density as a
``Lorentzian''\citep{howard88,hutter93,roters96,levy02,florin95}, but
there is dissagreement on what this means.  The classic Lorentzian
function is\citep{mathworld-lorentzian}
\begin{equation}
  L(x) = \frac{1}{\pi}\frac{\frac{1}{2}\Gamma}
                           {(x-x_0)^2 + \p({\frac{1}{2}\Gamma})^2} \;,
  \label{eq:lorentzian}
\end{equation}
where $x_0$ sets the center and $\Gamma$ sets the width of the curve.
However, the correct \PSD\ for a damped harmonic oscillator in a white
noise bath is given by \cref{eq:psd-Vp}\citep{burnham03,benedetti12}.

These formulas are fundamentally different.

For example, the slope of \cref{eq:psd-Vp} is zero at $f=0$, as we can
see by using the chain rule repeatedly,
\begin{align}
  \deriv{f}{\PSD_f}
    &= \deriv{f}{}\p({\frac{G_{1f}}{(f_0^2-f^2)^2 + \beta_f^2 f^2}})
    = \frac{-G_{1f}}{\p({(f_0^2-f^2)^2 + \beta_f^2 f^2})^2}
      \deriv{f}{}\p({(f_0^2-f^2)^2 + \beta_f^2 f^2}) \\
    &= \frac{-G_{1f}}{\p({(f_0^2-f^2)^2 + \beta_f^2 f^2})^2}
      \p({2(f_0^2-f^2)\deriv{f}{}(f_0^2 - f^2) + 2\beta_f^2 f}) \\
    &= \frac{-G_{1f}}{\p({(f_0^2-f^2)^2 + \beta_f^2 f^2})^2}
      \p({-4f(f_0^2-f^2) + 2\beta_f^2 f}) \\
    &= \frac{2G_{1f}f}{\p({(f_0^2-f^2)^2 + \beta_f^2 f^2})^2}
      \p({2(f_0^2-f^2) - \beta_f^2})
    \label{eq:model-psd-df} \\
  \left.\deriv{f}{\PSD_f}\right|_{f=0} &= 0 \;.
    \label{eq:model-psd-df-zero}
\end{align}
On the other hand, the slope of \cref{eq:lorentzian} is only zero at
the peak (where $x=x_0$).
\begin{align}
  \deriv{x}{L(x)}
    &= \frac{1}{\pi}\frac{\frac{-1}{2}\Gamma}
                         {\p({(x-x_0)^2 + \p({\frac{1}{2}\Gamma})^2})^2}
         \cdot \deriv{x}{}\p({(x-x_0)^2 + \p({\frac{1}{2}\Gamma})^2}) \\
    &= \frac{1}{\pi}\frac{\frac{-1}{2}\Gamma}
                         {\p({(x-x_0)^2 + \p({\frac{1}{2}\Gamma})^2})^2}
         \cdot 2 (x-x_0) \\
    &= \frac{1}{\pi}\frac{-\Gamma (x-x_0)}
                         {\p({(x-x_0)^2 + \p({\frac{1}{2}\Gamma})^2})^2}
  \label{eq:lorentzian-dx}
\end{align}

It is unclear whether the ``Lorentzian'' references are due to
uncertainty about the definition of the Lorentzian or to the fact that
the two equations have similar behavior near the
peak.  \citet{florin95} likely \emph{are} using \cref{eq:lorentzian},
as the slope of the fitted \PSD\ in their \fref{figure}{2}, has a
slope at $f=0$.  If they were using \cref{eq:psd-Vp}, the derivative
would have been zero (\cref{eq:model-psd-df-zero}).

We have at least two models in use, one likely the ``Lorentzian''
(\cref{eq:lorentzian}) and one that's not.  Perhaps researchers
claiming to use the ``Lorentzian'' are consistently
using \cref{eq:lorentzian}?  There are at least two
counterexamples---\citet{roters96,benedetti12}---with solid
derivations of
\cref{eq:DHO-psd} which they then refer to as the ``Lorentzian''.
Which formula are the remaining ``Lorentzian'' fitters using?  What
about groups that only reference their method as ``thermal
calibration'' without specifying a \PSD\ model?  In order to avoid any
uncertainty, we leave \cref{eq:psd-Vp} unnamed.  I encourage future
researchers to explicitly list the model they use, ideally by citing
their associated open source calibration package.
