\begin{preamble}

\iffinal{}{\newpage}
\begin{DUTdedications}
%
\vspace*{\fill}
%
\begin{center}
\setstretch{2}
\begin{minipage}{8 cm}
\begin{center}
\hrulefill\\
This thesis is dedicated to Jennifer McLamb. Her unwavering support and understanding was the difference between success and failure. I love you so much, and want you to know that this thesis was only possible thanks to you. 

%\hrulefill\hspace{0.2cm} \leafNE \hspace{0.2cm} \hrulefill
\hrulefill
\vspace{6em}
\end{center}
\end{minipage}
\end{center} 
%
\vspace*{\fill}
%
\end{DUTdedications} 
\iffinal{}{\newpage}

\begin{acknowledgments}
  \iffinal{}{\setstretch{1.5}}
 I would first like to thank my advisors, Dr. Jelena Maricic and Dr. Charles Lane, whose support and guidance allowed for the completion of this work. I would further thank the support staff at Argonne National Laboratory, particularly Frank Skrecz and Victor Guarino, both of whom went more than the extra mile to ensure the completion of the Articulated Arm, a project which had more than a few setbacks and false starts. 

I would also like to acknowledge the support and advice for analysis provided by Dr. Michelle Dolinski, whose assistance was crucial to the paraphoton section. I would further like to recognize members of the Double Chooz collaboration who provided software advice and a helping hand, particularly Mariano, Bernd, and Lee. I would like to thank my many editors, particularly Erica Smith and William Damon. They reduced the level of typographical and grammatical mess from all-consuming to manageable. Finally, a special thank-you to my mom and the Simmons, who were always willing to listen to me complain about my research. 
 
  \end{acknowledgments}
  \iffinal{}{\newpage}

  \listoffigures 
  \iffinal{}{\newpage}

  \tableofcontents 
  \iffinal{}{\newpage}

  \begin{abstract}
  \iffinal{}{\setstretch{1.5}}
  \setstretch{1.5}
 This thesis details the design, development, and use of a novel calibration system for the Double Chooz experiment, the Articulated Arm. The Articulated Arm is capable of making radioactive source deployments throughout the volume of the Double Chooz Neutrino Target with a precision of $<$ 1 cm. The Articulated Arm is a powerful asset, able to validate detector response through direct comparison of on-axis and off-axis points. 
 
 This thesis also details a search for paraphoton signals in the Double Chooz 3rd publication dataset. The paraphoton is a light, electrically neutral, axion-like particle, generated from adding a new $U(1)$ symmetry of Baryon number - Lepton Number, with a Lagrangian of the form: $\mathcal{L} = -\frac{1}{4}F^{\mu\nu}F_{\mu\nu} - \frac{1}{4}B^{\mu\nu}B_{\mu\nu} -\frac{1}{2} \chi F^{\mu\nu}B_{\mu \nu} + \frac{1}{2} m_{\gamma'}^2 B_\mu B^\mu$. We search for decays of these particles inside the Double Chooz Far Detector using a rate-based method. We observed no events in excess of background, giving a new laboratory limit on the photon-paraphoton mixing angle $\chi$ at the 95$\%$ confidence level for paraphoton masses between 26 and 30 keV. Our limiting value was $\chi > 4.23 \times 10^{-3}$ for $m_{para} =26$ keV paraphotons.
 
  \end{abstract} 

	\iffinal{}{\newpage}

\end{preamble}
