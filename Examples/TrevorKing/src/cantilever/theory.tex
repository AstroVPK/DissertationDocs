\section{Theoretical background}
\label{sec:cantilever:theory}

Understanding a protein's free energy landscape is important to
effectively model protein folding and unfolding behavior.  Force
spectroscopy has been a useful technique for exploring these free
energy landscapes and those of the related field of ligand-receptor
kinetics.  In force spectroscopy with the atomic force microscope
(AFM), it is common practice to use spring constants in the range of
$50\U{pN/nm}$, but the effect of the cantilever itself on the free
energy landscape is generally ignored.  However, in AFM
biotin-streptavidin unbinding experiments, \citet{walton08}
demonstrated a surprisingly strong effect on unbinding force due to
cantilever stiffness.  The unbinding force approximately doubled due
to a change from a $35\U{pN/nm}$ cantilever to a $58\U{pN/nm}$
cantilever.  Alarmed by the magnitude of the shift, we repeated their
experiment on octomeric I27 to determine the magnitude for our
mechanical protein unfolding experiments.

\begin{figure}
  \asyinclude{figures/schematic/landscape-cant}
  \caption{Energy landscape of a protein being stretched by an
    external force.  For a discussion of methods for calculating
    unfolding rates from such a landscape, see
    \cref{sec:sawsim:rate}.\label{fig:landscape:cantilever}}
\end{figure}

The presence of attached linkers and cantilevers alters the free
energy landscape.  Tension in the linkers favors domain unfolding, but
that tension is not necessarily independent of the unfolding reaction
coordinate.  For sufficiently stiff cantilevers and linkers, even the
small extension of the domain as it shifts from its bound to
transition state noticeably reduces the effective tension.  Assuming
the bound and transition state extensions are relatively independent
of the applied tension, the energy of the transition state will be
\begin{align*}
  E_b(f) &= E_b(f=0) - \int_0^{\Delta x} f(x) dx \\
         &= E_b(f=0) - \int_0^{\Delta x} [f(x=0) - \kappa x] dx \\
         &= E_b(f=0) - f(x=0) \Delta x + \frac{1}{2}\kappa \Delta x^2 \;,
\end{align*}
where $\kappa$ is the effective linker spring constant for that
tension.  The Bell-model unfolding rate is thus
\begin{align*}
  k(f) &= k_0 \exp{\frac{f\Delta x - \frac{1}{2}\kappa \Delta x^2}{k_B T}} \;,
\end{align*}
and stiffer linkers will increase the mean unfolding force.

Unfolded I27 domains can be well-modeled as wormlike chains (WLCs,
\cref{sec:sawsim:tension:wlc})\citep{carrion-vazquez99b}, where $p
\approx 4\U{\AA}$ is the persistence length, and $L \approx 28\U{nm}$
is the contour length of the unfolded domain.  Obviously effective
stiffness of an unfolded I27 domain is highly dependent on the
unfolding force, and for tensions $\sim 280\U{pN}$ is $\sim
190\U{pN/nm}$.  This is within a factor of four of common cantilever
spring constants, so cantilever stiffness drives the effective spring
constant for the first four domains, after which point I27 stiffness
takes the lead.

\begin{figure}
  \begin{tikzpicture}
    % Inspired by Florian Hollandt's RNA codons
    %   http://www.texample.net/tikz/examples/rna-codons-table/
    \tikzstyle{every node}=[inner sep=1.7pt,anchor=center]
    \tikzstyle{doublebond}=[double distance=3pt]
    \node (Ca) at (0,0) {};
    \node (Cb) at ($(Ca)+(0:1)$)  {};
    \node (Na) at ($(Cb)+(60:1)$) {N};
    \node (Cc) at ($(Na)+(0:1)$)  {C};
    \node (Nb) at ($(Cc)+(0:1)$)  {N};
    \node (Cd) at ($(Nb)+(60:1)$) {};
    \node (Ce) at ($(Cd)+(0:1)$)  {};
    \node (Cf) at ($(Ce)+(60:1)$) {};
    \draw (Ca.center) -- (Cb.center) -- (Na);
    \draw[doublebond] (Na) -- (Cc) -- (Nb);
    \draw (Nb) -- (Cd.center) -- (Ce.center) -- (Cf.center);
  \end{tikzpicture}%
  \caption{1-Ethyl-3-(3-dimethylaminopropyl)carbodiimide (EDC), a
    short, stiff linker which has been used to bind proteins to gold
    surfaces\cite{lee05}.\label{fig:EDC}}
\end{figure}
