\section{History}
\label{sec:hooke:history}

\Hooke\ was originally developed by \citet{sandal09} working at the
University of Bologna.  It was actively developed until the paper came
out, after which development became more sporadic.  This was partly
because \Hooke\ worked well enough for the original authors and partly
because some of the developers had graduated and moved on to other
fields\footnote{
  Developer turnover may seem like a good reason to avoid open source
  software.  Why use something when its developers may not stay around
  to support it?  This argument may make sense if you're comparing
  open source and commercial packages, but it makes less sense if
  you're comparing existing open source packages to hypothetical
  in-house software.  Why \emph{not} use something, if it's free and
  already exists?  Figuring out someone else's software is often much
  more efficient than writing your own tool from
  scratch\footnotemark{}.
}\footnotetext{
  \ldots says the person who threw out the existing implementation and
  rewrote the control stack from scratch ;).  I'm ok with starting
  over if the existing project is not maintainable, but realize that
  you're probably biting off a lot of work.
}.

Before discovering \Hooke in 2010, I had been using a series of fairly
site-specific scripts to post-process my unfolding data.  Excited by
the existence of a published, open source post-processing framework, I
dropped my scripts and started working on \Hooke.  Other open-source
tools for post-processing SMFS data exist, but they are based on
closed sorce tools\citep{kuhn05,aioanei11} and some are no longer
being developed\citep{kuhn05}.  There are also some completely closed
source tools such as \citetalias{punias}\citep{carl08} and JPK's
\citetalias{force-robot}\citep{struckmeier08}.  Other work along this
line exists, but source code is unavailable\citep{andreopoulos11}.

\Hooke\ supports a wide range of input file formats via \emph{drivers}
(\cref{sec:hooke:drivers}), but when I began working on the project,
there wasn't a clear interface between the drivers and processing
\emph{plugins} (\cref{sec:hooke:plugins}).  I cleaned up this
interface as part of a general refactoring, fixing a number of plugins
that relied on obscure internals in particular driver code.  My
refactoring removed these leaky abstractions, and now every analysis
plugin works with every data driver.

\Hooke\ has been designed with an eye to modular flexibility, with a
command line interface (CLI) and graphical user interface (GUI).
However, as with drivers and plugins, the initial abstractions were
leaky.  I added a generic argument interface for analysis plugins, and
taught the interfaces how to handle the generic argument types
(\cref{sec:hooke:ui}).  With this framework, new analysis plugins are
automatically supported by both the CLI and the GUI.  As a side effect
of this reorganization, the GUI (which had previously been a display
for the CLI) became truly interactive.  The interactive portions of
the GUI owe a large debt to earlier work by Rolf Schmidt et al.~from
the Centre for NanoScience Research at Concordia University.
%
\nomenclature[text ]{UI}{User interface.  What a user uses to interact
  with a software package (\cf~GUI and CLI).}
\nomenclature[text ]{CLI}{Command line interface.  A textual computing
  environtment, where the user controls execution by typing commands
  at a prompt (\cf~GUI and UI).}
\nomenclature[text ]{GUI}{Graphical user interface.  A graphical
  computing environment, where the user controls execution through
  primarily through mouse clicks and interactive menus and widgets
  (\cf~CLI and UI).}
