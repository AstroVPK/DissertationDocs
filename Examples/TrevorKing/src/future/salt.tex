\section{Salt}
\label{sec:future:salt}

As expected\citep{chauhan97,itkin11,zidar11}, increasing the ionic
strength of the buffer did significantly decrease the unfolding force
(folding stability) of I27.  For labs with strong gene-splicing
capability, it would be interesting to replace the glutamic acids
involved in the major bonding (\cref{fig:I27:H-bonds}) with
alternative groups to gauge the specificity of the effect.  For
example, glutamine is identical to glutamic acid, except that it has a
hydroxyl group (OH) in the side-chain where glutamic acid has an
amine group (NH\textsubscript{2}).  This gives glutamine and glutamic
acid similar steric properties, but very different chemistry.

While the statistics are strong for the two concentrations we tested
(standard PBS and PBS with an additional $0.5\U{M}$ \CaCl), it would
be useful to study destabilization scaling over a range of
concentrations.  Carrying out these experiments over a range of
pulling speeds with additional force clamp experiments would also
increase confidence in the kinetic models used to summarize the data.
SMFS is a low-throughput technique, so such an exponential increase in
assembled data would be much easier with more reliable hardware.
