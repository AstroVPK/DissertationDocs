\chapter*{Preface}

Plato, in the \emph{Phaedrus}, likened the human character to a chariot driven by two horses, one ill-tempted and lusty, and the other well-behaved and virtuous. In order for the chariot to move forward, the wildness of the evil horse had to be mastered, while the strength of the good horse must be nurtured; both must pull together in harmony. The field of particle physics is governed by two impulses as well ---- theory and experiment --- and both must pull in unison for science to advance. 

The history of particle physics is full of examples of this nature: theory guiding experiment, and experiment, in turn, informing theory, and, from time to time, discovering new and unusual trails that lead in all sorts of directions. Particle physics is also full of examples of the two horses not working together in harmony. Experiment, without guiding theory, tends to move very slowly indeed, creating a great deal of heat for little light, and getting stuck in blind alleys. Theory, unchained from experimental data, tends to blind onlookers with the profusion of alternative models, while at the same time producing almost nothing fruitful. 

This thesis represents a fusion of theory and experiment. In it are combined the powerful theoretical insights of the latest in beyond the standard model physics with new experiments and calibration devices. Despite this pedigree, it does not claim any great discovery, merely pushing the frontiers of knowledge back in some small ways.

Tolstoy uses the metaphor of integration in \emph{War and Peace}, the adding up of infinitesimals that makes up something large. While no work is truly infinitesimal, the point stands; there can be no Napoleon without his army. It is the results that point towards entirely new physics that tend to be romanticized and remembered. However, most of the time, the work of scientists is what Kuhn called "normal science" in \emph{Structure of Scientific Revolutions}, the everyday confirmation or rejection of theory, adding detail and texture to the broad brushstrokes that are the starting points of new physics. 

This work is vital, particularly so in fields which exist on the cusp of something new, as in neutrino physics. It is a seeming paradox; there is so much unexplained and unknown about the neutrino that even new discoveries, like the values for the mixing angles in the PMNS matrix, seem humdrum. This war of attrition, the slow wearing down of the unknown, lacks the drama of the eureka moment, the brilliant tactical maneuver that changes the course of events, but it only by these small discoveries, purchased at great cost, that science advances.
