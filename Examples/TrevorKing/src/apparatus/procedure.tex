\section{Mechanical unfolding experiments}
\label{sec:procedure}

% AFM unfolding procedure
In a mechanical unfolding experiment, a protein polymer is tethered
between two surfaces: a flat substrate and an AFM tip.  The polymer is
stretched by increasing the separation between the two surfaces
(\cref{fig:unfolding-schematic}).  The most common mode is the
constant speed experiment in which the substrate surface is moved away
from the tip at a uniform rate.  The tethering surfaces, \ie, the AFM
tip and the substrate, have much larger radii of curvature than the
dimensions of single domain globular proteins that are normally used
for folding studies.  This causes difficulties in manipulating
individual protein molecules because nonspecific interactions between
the AFM tip and the substrate may be stronger than the forces required
to unfold the protein when the surfaces are a few nanometers apart.
To circumvent these difficulties, globular protein molecules are
linked into polymers, which are then used in the AFM
studies\citep{carrion-vazquez99a,chyan04,carrion-vazquez03}.  When
such a polymer is pulled from its ends, each protein molecule feels
the externally applied force, which increases the probability of
unfolding by reducing the free energy barrier between the native and
unfolded states.  The unfolding of one molecule in the polymer causes
a sudden lengthening of the polymer chain, which reduces the force on
each protein molecule and prevents another unfolding event from
occurring immediately.  The force versus extension relationship, or
\emph{force curve}, shows a typical sawtooth pattern
(\cref{fig:expt-sawtooth}), where each peak corresponds to the
unfolding of a single protein domain in the polymer.  Therefore, the
individual unfolding events are separated from each other in space and
time, allowing single molecule resolution despite the use of
multi-domain test proteins.
%
\nomenclature[text ]{force curve}{Or force--distance curve.
  Cantilever-force versus piezo extension data aquired during a force
  spectroscopy experiment (\cref{fig:expt-sawtooth}).}

\begin{figure}
  \begin{center}
  \subfloat[][]{\asyinclude{figures/schematic/unfolding}%
    \label{fig:unfolding-schematic}}
  \hspace{.25in}%
  \subfloat[][]{\asyinclude{figures/expt-sawtooth/expt-sawtooth}%
    \label{fig:expt-sawtooth}}
  % Possibly use carrion-vazquez00 figure 2 to show scale of afm tip
  \caption{\protect\subref{fig:unfolding-schematic} Schematic of the
    experimental setup for mechanical unfolding of proteins using an
    AFM (not to scale).  An experiment starts with the tip in contact
    with the substrate surface, which is then moved away from the tip
    at a constant speed.  $x_t$ is the distance traveled by the
    substrate, $x_c$ is the cantilever deflection, $x_u$ is the
    extension of the unfolded polymer, and $x_f=x_{f1}+x_{f2}$ is the
    extension of the folded polymer.
    \protect\subref{fig:expt-sawtooth} An experimental force curve
    from stretching a ubiquitin polymer (\tikzline{red}) with the
    rising parts of the peaks fitted to the WLC\index{WLC} model
    (\tikzline{blue}, \cref{sec:sawsim:tension:wlc})\citep{chyan04}.
    The pulling speed used was $1\U{$\mu$m/s}$.  The irregular
    features at the beginning of the curve are due to nonspecific
    interactions between the tip and the substrate surface, and the
    last high force peak is caused by the detachment of the polymer
    from the tip or the substrate surface.  Note that the abscissa is
    the extension of the protein chain
    $x_t-x_c$.\label{fig:procedure}}
  \end{center}
\end{figure}
