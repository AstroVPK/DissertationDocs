\begin{preamble}

\iffinal{}{\newpage}

\begin{DUTdedications}
\begin{center}
%\hrulefill\\
This thesis is dedicated to my family \\
whose love and support made this work possible\\
%\hrulefill
\end{center}
\end{DUTdedications}

\iffinal{}{\newpage}

\begin{acknowledgments}
%Over the past six years I have received support and encouragement form many people.  
This work would not have been possible without the support of my advisor, Dr. Gordon Richards. His guidance helped to shape and provided much needed focus to my work.  I would like to thank my dissertation committee of Dr. Michael Vogeley, Dr. David Goldberg, Dr. Adam Lidz, and Dr. Shyamalendu Bose for their support and insight throughout my research.

I would also like to thank my fellow graduate students for all the help and support they provided. During data analysis Micholas Smith and David Lioi spent countless hours listening to me talk about my research, helping me flesh out my ideas. And a special thanks to Erica Smith for proofreading all of my work and for encouraging me throughout my graduate work. 
\end{acknowledgments}

\iffinal{}{\newpage}

\tableofcontents 
\iffinal{}{\newpage}

\listoftables
\iffinal{}{\newpage}

\listoffigures 
\iffinal{}{\newpage}

\begin{abstract}
%\ifdaring{\setstretch{1.3}}{}
\ifdaring{\setstretch{1.3}}{\setstretch{1.6}}

We explore the mid-infrared (mid-IR) through ultraviolet (UV) spectral energy distributions (SEDs) of 119,652 luminous type 1 quasars with $0.064<z<5.46$ using mid-IR data from {\em Spitzer} and {\em WISE}, near-infrared data from 2MASS and UKIDSS, optical data from SDSS and UV data from {\em GALEX}.  The mean SED requires a bolometric correction of \bctwofive$=2.75\pm0.40$ using the integrated light from \onemum--\twokev. 
We investigate the mean SED dependence on various parameters, particularly the UV luminosity for quasars with $0.5\lesssim z\lesssim3$.
Low-luminosity SEDs exhibit a bluer far-UV spectral slope, a redder optical continuum, and less hot dust. 
Our work suggests that lower-luminosity quasars may require an extra continuum component in the unseen extreme-UV that is weak in high-luminosity quasars. As such, we consider four possible models and explore the resulting BCs.

Taking a subset of $\sim35,000$ uniformly selected quasars we explore their extinction/reddening in order to better understand their {\em intrinsic} SEDs. 
Using optical--UV photometry, we isolate outliers in the color distribution and find them well described by an SMC-like reddening law.
A hierarchical Bayesian model was used to find distributions of powerlaw indices and $\ebv$ consistent with both the broad absorption line (BAL) and non-BAL samples. 2.5\% (13\%) of the non-BAL (BAL) sample are shown to be consistent with $\ebv>0.1$ and 0.1\% (1.3\%) with $\ebv>0.2$. Simulations show both populations of quasars are intrinsically bluer than the mean composite, with a mean spectral index ($\alpha_{\lambda}$) of -1.79 (-1.83).  
The emission and absorption-line properties of both samples showed that quasars with intrinsically red continua have weaker Balmer lines and stronger ionizing spectral lines, the latter indicating a harder continuum in the extreme-UV.

%\ion{C}{4}, \ion{He}{2}, \ion{C}{3}], and [\ion{O}{2}], while the  \ion{Mg}{2}, H$\gamma$, H$\beta$, and H$\alpha$ lines are weaker.

Applying corrections for associated dust, we better determine the {\em intrinsic} SEDs and true BCs for our uniformly selected subsample. The SEDs with the most dust extinction are intrinsically brighter and showed more dust emission near $\sim 10\,\mu$m than the SEDs with less extinction. The bluer SEDs have more hot dust emission and higher BCs, consistent with having hotter accretions disks and/or being viewed closer to edge-on.
Mean SEDs were also made based on the black hole mass ($\mbh$) and the Eddington fraction ($\lf$).  Quasars with large $\lf$ and/or large $\mbh$ have more hot dust and a bluer optical continua, both consistent with a hotter accretion disk.

%We applied extinction corrections for associated dust to the observed quasar SEDs to better determine the {\emph{intrinsic}} SEDs and true BCs.  
%We found bluer quasars have more hot dust and higher BCs.
%SEDs with more extinction showed more total dust emission and smaller BCs than the SEDs with less extinction.
%Mean SEDs were made based on the mass of the central black hole ($\mbh$) and the Eddington fraction ($\lf$).  Quasars with large $\lf$ and/or large $\mbh$ have more hot dust and bluer optical continua while those with large $\mbh$ also show an additional bump in the extreme-UV.

\end{abstract}

\iffinal{}{\newpage}

\end{preamble}