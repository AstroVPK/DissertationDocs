\chapter{Final Remarks}
%
In this thesis, we have developed several new methods and models, each an attempt to provide insight into the role of entropy in the protein folding process. These studies and others like it, are paramount to a full understanding of protein models. The breadth of the research chapters, depletion forces, crowding, aggregation, and macrostate clustering, emphasizes the fact that protein folding process is still very much an open problem. 

Each of the four research chapters highlight particular suggestions for model improvements, but there are a few overarching directions that seem to be common to all of them. First and foremost, we can't underestimate the need for more comparisons with experimental data. While Chapters \ref{chap:entropic_force} and \ref{chap:WL_crowding} were designed around previously measured data, the model theories have only been extrapolated to different systems, but never tested. The models in Chapters \ref{chap:potts_aggregation} and \ref{chap:clustering_kinetics} have yet to be applied and tested by experimental data outside of a few small systems. This presents an opportunity for both new experimental measurements and extensions of the models to existing data.

In addition to the experimental connection, many of the models present a simplified treatment of the enthalpic terms. This was motivated by the idea to study the effects of entropy exclusively. Many of the calculations however, would have been much more difficult if a more realistic potential had been used. While we feel that a simplified Hamiltonian can capture the essence of the entropic forces, it does present an incomplete picture of physical reality. While the conclusions may remain the same, the supporting evidence could only be strengthened by a more realistic potential. The smaller more focused studies presented here serve an important role as building blocks for a more complete description of the biophysical process. 

It is quite possible to combine molecular dynamics simulations with some of the  ideas presented here. A first step would be the replacement of the $\Go$-like or \chem{H P C} model with the empirically based residue to residue contact matrix defined by Miyazawa and Jernigan.\cite{miyazawa_rl:empirical_1999} In addition the lattice restriction could be removed, or at very least, extended to more complicated geometries. Independently, the macrostate clustering algorithm can serve as a starting point for molecular dynamic simulations, guiding the research to the salient points in conformational space. The time-series method in Chapter \ref{chap:clustering_kinetics} can be applied to the multitude of molecular dynamics trajectories from other numerical experiments.

Each day, new observations are being made that seem to highlight the importance of an \textit{in vivo} model; crowding and aggregation processes are often incomplete without them. Our study here was an attempt to simplify the process at various levels of complexity, so as to give a deeper understanding of the protein folding problem.
