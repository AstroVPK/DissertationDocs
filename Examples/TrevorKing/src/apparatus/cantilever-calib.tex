\section{Cantilever spring constant calibration}
\label{sec:cantilever-calib:intro}

In order to measure forces accurately with an AFM, it is important to
measure the cantilever spring constant $\kappa$\index{$\kappa$}.  The
force exerted on the cantilever can then be deduced from its
deflection via Hooke's law,\index{Hooke's law}
\begin{equation}
  F=-\kappa x \;,  \label{eq:hooke}
\end{equation}
where $x$ is the perpendicular displacement of the cantilever tip
($x_c$ in \cref{fig:unfolding-schematic}).
%
\nomenclature[sr ]{$F$}{Force (newtons).}
\nomenclature[sg k ]{$\kappa$}{Spring constant (newtons per meter).}
\nomenclature[sr ]{$x$}{Displacement (meters).}

The basic idea is to use the equipartition theorem, which gives the
thermal energy per degree of freedom.  For a simple harmonic
oscillator, the only degree of freedom is $x$, so we
have\citep{hutter93}
\begin{equation}
  \frac{1}{2} \kappa \avg{x^2} = \frac{1}{2} k_BT \;, \label{eq:equipart}
\end{equation}
where $k_B$ is Boltzmann's constant, $T$ is the absolute temperature,
and $\avg{x^2}$ is the average value of $x^2$ measured
over a long time interval.
%
\nomenclature[sr ]{$k_B$}{Boltzmann's constant,
  $k_B = 1.380 65\E{-23}\U{J/K}$\citep{codata-boltzmann}.}
\nomenclature[sr ]{$T$}{Absolute temperature (Kelvin).}
\nomenclature[o  ]{$\avg{s(t)}$}{Mean (expectation value) of a
  time-series $s(t)$
  \begin{equation}
    \avg{A} \equiv \iLimT{A} \;.
  \end{equation}}
\nomenclature[o  ]{$\equiv$}{Defined as (\ie\ equivalent to).}

To calculate the spring constant $\kappa$ using \cref{eq:equipart}, we
need to measure the buffer temperature $T$ and the thermal vibration
variance $\avg{x^2}$.  We measure the temperature with a thermocouple
inserted into the AFM fluid cell, and we measure the thermal vibration
by monitoring the cantilever during thermal oscillation when it is far
from the substrate surface.

The raw cantilever deflection data will have sources of noise that are
not due to the cantilever's thermal vibration (e.g.~electronic noise
in the detector).  To avoid biasing $\kappa$, there is a fairly
elaborate theory behind extracting $\avg{x^2}$.  For more detail, see
\cref{sec:calibcant}, where I discuss the $\avg{x^2}$ extraction in
detail and present my open source \calibcant\ tool for automated
cantilever calibration.
