\section{Auxiliary packages}
\label{sec:pyafm:auxiliary}

The previous section covered the core of the experiment stack
(\cref{sec:pyafm:stack}), but skipped over some of the more peripheral
packages.

\subsection{h5config}
\label{sec:pyafm:h5config}

The \hFconfig\ package makes it easy to save and load configuration
classes from disk.  After populating base configuration classes with
parameters (\cref{fig:h5config}), \hFconfig\ automatically generates
\citetalias{hdf5} and \citetalias{yaml} backends for saving and
loading that class.

\begin{figure}
  \begin{center}
\begin{minted}[mathescape]{python}
import h5config.config as _config

class AxisConfig (_config.Config):
    "Configure a single piezo axis"
    settings = [
        _config.FloatSetting(
            name='gain',
            help=(
                'Volts applied at piezo per volt output from the DAQ card '
                '(e.g. if your DAQ output is amplified before driving the '
                'piezo),')),
        _config.FloatSetting(
            name='sensitivity',
            help='Meters of piezo deflection per volt applied to the piezo.'),
        # $\ldots$
        _config.ConfigSetting(
            name='channel',
            help='Configure the underlying DAC channel.',
            config_class=OutputChannelConfig,
            default=None),
        # $\ldots$
        ]
\end{minted}
    \caption{Portions of the configuration class for a single piezo
      axis (from \pypiezo, \cref{sec:pyafm:pypiezo}).  The more
      generic analog output channel configuration is nested under the
      \imint{python}|channel| setting.\label{fig:h5config}}
  \end{center}
\end{figure}

Basic configuration types include booleans, integers, floating point
numbers, enumerated choices, and freeform text.  There is also support
for lists of these basic types (e.g.~lists of integers).  The key
feature is nesting configuration classes.  This means that your higher
level tools can have their own configuration settings and also include
the configuration settings for their lower level components.  For
example, the piezo axis configuration given in \cref{fig:h5config}
contains configuration settings specific to piezo axes, and it also
contains a reference to the configuration settings for a generic
analog output channel.  The piezo axis code doesn't need to know what
the analog output channel configuration settings are, those are
defined somewhere else.

The nesting continues all the way up the stack, to the
\unfoldprotein\ configuration.  This means that a single file
(\imint{console}|~/.config/unfold_protein.yaml|) contains every
configurable setting required for the whole experiment in an
easy-to-edit text format.  Adding additional configuration settings at
any level of the experiment stack is just a matter of adjusting a
single \imint{python}|h5config.config.Config| subclass the
corresponding entry in the configuration file.  There is no need to
adjust the higher level code, the new setting is passed down the stack
to its point of use automatically.

Besides making it easy to configure your experiment, \hFconfig\ also
makes it easy to save the configuration alongside your data.  The
section of \unfoldprotein\ that writes the whole configuration stack
into the per-pull HDF5 file is only four lines long.  This makes
post-processing much easier, because almost every setting needed to
analyze the data is already stored in the data file (the only missing
values are those that you did not need during the experiment control
phase).

\subsection{stepper}
\label{sec:pyafm:stepper}

Because of thermal drift and mechanical instability, the distance
between the tip and the surface changes significantly over time.  When
the distance change exceeds the range of the piezo scanner, the
stepper motor must be engaged to reposition the AFM tip relative to
the sample.  The earlier LabVIEW software (\cref{sec:labview}) lacked
the ability to control the motor on its own, so it would pause roughly
every half hour and prompt the operator to make the necessary manual
adjustments.  Automatic motor control allows the system to run longer
without interrupts, facilitating the collection of large data sets.

The \stepper\ package provides Python control of stepper
motors\cite{jones95}.  The package is mostly concerned with the
maintenance of internal motor state:

\begin{description}
  \item[position] is a half-step counter that records the current
    motor position.
  \item[full step] selects full or half stepping.
  \item[logic] selects active high or active low operation.
  \item[delay] sets the time delay between steps in seconds, in case
    the motor response is slower than the digital output driver.
  \item[step size] approximates the step size in meters.
  \item[backlash] estimates the drive chain backlash in half-steps.
\end{description}

Actualizing the motor control signal is left up to the caller, in this
case \pyafm.

We verified the stability and reproducibility of the microscopic
movement of the motor by making several approach-retreat cycles from
the surface of $\sim$70 steps which resulted in the data displayed in
\cref{fig:stepper:backlash}.  We also measured the distance the
surface moved with every step by determining the change in deflection
voltage as a function of peizo position as we stepped the AFM tip
closer the the surface.  Our stepsize data is displayed in
\cref{fig:stepper:step-size}.

The motor is very consistent when approaching the surface, which
indicates that our control software is operating correctly.  However,
the motor exhibits some hysteretic behavior on a scale of $\sim$46
steps, which is almost certainly due to \emph{backlash}, or slack in
the motor--surface coupling machinery.  The first 46 steps in a new
direction take the slack out of the coupling, and further steps move
the tip relative to the surface.  The problem can be avoided entirely
by simply replacing ``backwards motion by one step'' with ``backwards
motion by 60 steps and forward motion by 59 steps''.

One issue raised by backlash is that it might be the source of some of
our surface drift, as the drive-chain relaxes towards some central
value and pulls the surface with it.  By oscillating into our eventual
position, we could perhaps settle the system at the beginning,
reducing the need for adjustments later on.  While this is not a
problem for the current unfolding experiments, it could be an issue
for longer unfolding-refolding experiments.

\begin{figure}
  \begin{center}
    \subfloat[][]{\label{fig:stepper:backlash}
      \asyinclude{figures/stepper/backlash}}
    \hspace{\stretch{1}}
    \subfloat[][]{\label{fig:stepper:step-size}
      \asyinclude{figures/stepper/step-size}}
    \caption{\protect\subref{fig:stepper:backlash} Stepper motor
      reproducibility, stability, and backlash.  The data are from a
      single continuous counterclockwise trace of 14 approach-retreat
      cycles.  The jump from about $(18, -6)$ to $(17, -0.4)$ is the
      snap-off effect, where short-range attractive interactions
      between the tip and the sample---due to surface wetting in
      air---require the tip to be actively pulled off surface.  Signal
      noise is comparable to that expected by drift.
      \protect\subref{fig:stepper:step-size} Motor step size
      calibration.  The stepper gradually stepped closer to the
      surface, feeling forward with the piezo after each step.
      Successive motor positions yield traces $a$, $b$, $c$, $d$, and
      $e$.  As the motor moves the sample closer, less piezo movement
      is required to approach to same deflection level.  The average
      spacing between the traces is roughly $170\U{nm}$.  Traces $c$
      and $d$ have regions of negative deflection because the tip no
      longer retracts far enough from the surface to break free of the
      snap-off effect.  There is no backlash because the data were
      taken during a single approach.\label{fig:stepper}}
  \end{center}
\end{figure}

\subsection{pypid}
\label{sec:pyafm:pypid}

The final component of the experiment control stack is \pypid, which
uses \citetalias{pymodbus} to communicate with a Melcor Series MTCA
Thermoelectric Cooler Controller\citep{melcor} over a serial line.
The controller monitors the fluid cell temperature with a
thermocouple, and reading temperatures from the controller is fairly
straightforward (\cref{fig:unfold-protein:unfolder}).  Temperature
control is via a Peltier device mounted underneath the sample surface
(\cref{fig:peltier}).

\begin{figure}
  \begin{center}
    \asyinclude{figures/schematic/peltier}
    \caption{A Peltier functions by applying a voltage to regions of
      p- and n-type semiconductor in series.  Conduction in n-type
      semiconductors is mainly through thermally excited electrons and
      in p-type semiconductors is mainly through thermally excited
      holes.  Applying a positive voltage as shown in this figure
      cools the sample by constantly pumping hot conductors in both
      semiconductors towards heat sink, which radiates the heat into
      the environment.  Reversing the applied voltage heats the
      surface.\label{fig:peltier}}
  \end{center}
\end{figure}

The controller tries to keep the measured temperature at the setpoint
temperature via a modified proportional-integral-derivative (PID)
feedback algorithm.  PID systems have been around for a
while\citep{ziegler42}, but finding appropriate feedback terms for
sensitive systems is not trivial.  There are a number of tuning
procedures which characterize the system by evaluating its response
under simpler driving conditions.  The \pypid\ package implements
Ziegler--Nichols' step response\citep{ziegler42}, bang-bang response,
and ultimate cycle response\citep{ziegler42} tuning rules, as well as
Cohen--Coon's\citep{cohen53} and Wang--Juang--Chan's\citep{wang95}
step response tuning rules\citep{astrom93}.
%
\nomenclature[text ]{PID}{Proportional-integral-derivative feedback.
  For a process value $p$, setpoint $p_0$, and manipulated variable
  $m$, the standard PID algorithm is
  \begin{align}
    m(t) &= K_p e(t) + K_i \integral{0}{t}{\tau}{e(\tau)}
            + K_d \deriv{t}{e(t)} \\
    e(t) &= p_0 - p \;,
  \end{align}
  where $e$ is the error function, $K_p$ is the proportional gain,
  $K_i$ is the integral gain, and $K_d$ is the derivative gain.}
