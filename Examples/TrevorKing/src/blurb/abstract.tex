\begin{abstract}
Single molecule force spectroscopy (SMFS) experiments provide an
experimental benchmark for testing simulated and theoretical
predictions of protein unfolding behavior.  Despite it use since
1997\citep{rief97a}, the labs currently engaged in SMFS use in-house
software and procedures for critical tasks such as cantilever
calibration and Monte Carlo unfolding simulation.  Besides wasting
developer time producing and maintaining redundant implementations,
the lack of transparency makes it more difficult to share data and
techniques between labs, which slows progress.  In some cases it can
also lead to ambiguity as to which of several similar approaches,
correction factors, etc.\ were used in a particular paper.
%
\nomenclature[text ]{SMFS}{Single molecule force spectroscopy.}

In this thesis, I introduce an SMFS sofware suite for cantilever
calibration (\calibcant), experiment control (\unfoldprotein),
analysis (\Hooke), and postprocessing (\sawsim) in the context of
velocity clamp unfolding of I27 octomers in buffers with varying
concentrations of
\CaCl\citep{calibcant,unfold-protein,sandal09,king10}.  All of the
tools are licensed under open source licenses, which allows SMFS
researchers to centralize future development.  Where possible, care
has been taken to keep these packages operating system (OS) agnostic.
The experiment logic in \unfoldprotein\ and \calibcant\ is still
nominally OS agnostic, but those packages depend on more fundamental
packages that control the physical hardware in use\citep{pyafm}.  At
the bottom of the physical-interface stack are the \Comedi\ drivers
from the Linux kernel\citep{comedi}.  Users running other operating
systems should be able to swap in analogous low level
physical-interface packages if Linux is not an option.
%
\nomenclature[text ]{OS}{Operating system.}

\end{abstract}
