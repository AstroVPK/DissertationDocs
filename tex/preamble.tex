\begin{preamble}

\iffinal{}{\newpage}
\begin{DUTdedications}
%
\vspace*{\fill}
%
\begin{center}
\setstretch{2}
\begin{minipage}{8 cm}
\begin{center}
\hrulefill\\
This thesis is dedicated to Cara Hoppe.\\ 
I respect her as my peer,\\
and cherish her as my wife.\\
Her love and support made this work possible. \\
%\hrulefill\hspace{0.2cm} \leafNE \hspace{0.2cm} \hrulefill
\hrulefill
\vspace{6em}
\end{center}
\end{minipage}
\end{center} 
%
\vspace*{\fill}
%
\end{DUTdedications} 
\iffinal{}{\newpage}

\begin{acknowledgments}
  \iffinal{}{\setstretch{1.5}}
The completion of this thesis could not have been done without the tireless support of my advisor, Dr. Jian-Min Yuan. I am deeply indebted for the support he has provided over the years. As a mentor, he helped me develop my skills as a scientist and honed my critical thinking. He is an endless source of new ideas and has taught me how to effectively communicate in the scientific world. 

I would also like to thank the physics department at Drexel for helping me prepare for the journey ahead. In particular, both Dr. Michel Vallieres and Dr. Robert Gilmore have been gracious enough to spend many afternoons discussing every interesting theory I've come across. Their insights and enthusiasm for physics and mathematics is inspiring. 

I would like to thank my family, my wife, Cara Hoppe, and my children, Hazel and Jackson Hoppe. They cheerfully remind me of the world outside and always bring a smile to my face. Finally, I would like to thank my uncle, Fred Stein, who introduced me to physics and all its wonders at an early age.
 
  \end{acknowledgments}
  \iffinal{}{\newpage}

  \listoffigures 
  \iffinal{}{\newpage}

  \tableofcontents 
  \iffinal{}{\newpage}

  \begin{abstract}
  \iffinal{}{\setstretch{1.5}}
  \setstretch{1.5}
A protein's ultimate function and activity is determined by the unique three-dimensional structure taken by the folding process. Protein malfunction due to misfolding is the culprit of many clinical disorders, such as abnormal protein aggregations. This leads to neurodegenerative disorders like Huntington's and Alzheimer's disease. We focus on a subset of the folding problem, exploring the role and effects of entropy on the process of protein folding. Four major concepts and models are developed and each pertains to a specific aspect of the folding process: entropic forces, conformational states under crowding, aggregation, and macrostate kinetics from microstate trajectories. 

The exclusive focus on entropy is well-suited for crowding studies, as many interactions are non-specific. We show how a stabilizing entropic force can arise purely from the motion of crowders in solution. In addition we are able to make a a quantitative prediction of the crowding effect with an implicit crowding approximation using an aspherical scaled-particle theory.

In order to investigate the effects of aggregation, we derive a new operator expansion method to solve the Ising/Potts model with external fields over an arbitrary graph. Here the external fields are representative of the entropic forces. We show that this method reduces the problem of calculating the partition function to the solution of recursion relations. 

Many of the methods employed are coarse-grained approximations. As such, it is useful to have a viable method for extracting macrostate information from time series data. We develop a method to cluster the microstates into physically meaningful macrostates by grouping similar relaxation times from a transition matrix.  

Overall, the studied topics allow us to understand deeper the complicated process involving proteins.
  \end{abstract} 

	\iffinal{}{\newpage}

\end{preamble}
