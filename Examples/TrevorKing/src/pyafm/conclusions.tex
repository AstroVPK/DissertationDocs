\section{Conclusions}
\label{sec:pyafm:conclusions}

Developing an open software stack for controlling single molecule
force spectroscopy is hard work, especially for scientists who lack
experience designing or managing moderately large software projects.
Coming into this project, I already had several years of experience
working with LabVIEW, but I had very little experience in other
languages and no formal training in project maintenance.  I spent the
first two years of my research project acquiring enough experience to
start making progress on a sustainable stack\footnote{%
  Since last summer I've been helping the
  \href{http://software-carpentry.org/}{Software Carpentry}
  project\citep{wilson06b} reach out to scientists (mostly graduate
  students) to provide boot camp introductions to software development
  and version control.  It's a chance to tell other folks what I wish
  I'd been told when I was starting out.
}, and I've spent the remaining time tuning this stack (and the
analysis software) while running experiments.

An open source stack allows collaborative development so that this
development cost can be shared between labs, as well as lowering the
barrier to entry for new labs entering the field.  Besides benefiting
SMFS groups, lower-level packages in the stack will be useful to a
wider audience (who can share the maintenance cost).  My existing
stack and future distributed maintenance will allow researchers to
focus on generating new science, instead of generating new software.

Besides development efficiencies, a common stack could provide a
benchmark for comparative analysis between experiments carried out by
different labs.  With every lab using in-house software and in-house
hardware, it's hard to judge the reliability or accuracy of the lab's
published research.  A common stack should include methods like those
used in \cref{sec:pyafm:discussion} to characterize and validate your
apparatus.  A common stack also provides a common file format for
experimental data, which makes it easier to share data and analysis
tools.

While there have been other attempts SMFS control which claim to be
open source, they have either been based on closed source
tools\citep{aioanei11} or have (critically) not actually published
their source\citep{materassi09}.  Both of these limitations make it
hard to realize the benefits of communal, open source development, and
the \pyafm\ stack suffers from neither.
