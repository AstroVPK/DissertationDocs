Understanding a protein's free energy landscape is important to
effectively model protein folding and unfolding behavior.  Force
spectroscopy has been a useful technique for exploring these free
energy landscapes and those of the related field of ligand-receptor
kinetics.  In force spectroscopy with the atomic force microscope
(AFM), it is common practice to use spring constants in the range of
$50\U{pN/nm}$, but the effect of the cantilever itself on the free
energy landscape is generally ignored.  However, in AFM
biotin-streptavidin unbinding experiments last year, Walton et al.\
demonstrated a surprisingly strong effect on unbinding force due to
cantilever stiffness\citep{walton08}.  The unbinding force
approximately doubled due to a change from a $35\U{pN/nm}$ cantilever
to a $58\U{pN/nm}$ cantilever.  Alarmed by the magnitude of the shift,
we repeated their experiment on octomeric I27 to determine the
magnitude for our mechanical protein unfolding experiments.
