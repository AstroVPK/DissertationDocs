% Why simulations?  And why are we leading with them?
Unfolding proteins by pulling on them with an AFM yields a series of
force curves (\cref{fig:expt-sawtooth}).  How do we get from the
unfolding curves to a deeper understanding of the unfolding physics?
In this chapter, I present my \sawsim\ simulator, which performs
discrete-state unfolding simulations using theoretical models of
protein unfolding.  By finding the models and parameters that best
reproduce the experimental data, we can find unforced unfolding rates
and other unfolded polymer distances that can be cross-checked in
chemical unfolding experiments and used to validate more detailed
molecular dynamics simulations.  The \sawsim\ simulations discussed
here are carried out after the unfolding experiments
(\cref{sec:pyafm}), cantilever calibration (\cref{sec:calibcant}), and
post-processing (\cref{sec:hooke}), but I discuss \sawsim\ first
because it provides the cleanest description of the underlying
physical processes.  After we cover the theory here, we will be better
prepared for the realistic complications discussed in the following
chapters.

% AFM unfolding analysis, what we'll do.
Much theoretical and computational work has been done in order to
extract information about the structural, kinetic, and energetic
properties of the protein molecules from the experimental data of
force-induced protein unfolding measurements.  Steered molecular
dynamics simulations\citep{lu98,lu99}, as well as calculations and
simulations using lattice\citep{klimov99,socci99} and off-lattice
models\citep{klimov00,li01}, have provided insights into structural
and energetic changes during force-induced protein unfolding.
However, these simulations often involve time scales that are orders
of magnitude smaller than those of the experiments
(\cref{sec:single-molecule}), and the parameters used in the
calculations are often neither experimentally controllable nor
measurable.
% (TODO: example parameters of each type).
As a result, a Monte Carlo simulation approach based on a simple
two-state kinetic model for the protein is usually used to analyze
data from mechanical unfolding experiments.  A comparison of the force
curves measured experimentally and those generated from simulations
can yield the unfolding rate constant of the protein in the absence of
force as well as the distance from the native state to the transition
state along the pulling direction.  The Monte Carlo simulation method
has been used since the first report of mechanical unfolding
experiments using the AFM%
\citep{rief97b,rief97a,rief98,carrion-vazquez99b,best02,zinober02,jollymore09},
but these previous implementations are neither fully described nor
publicly available.

To fill this gap, I developed the \sawsim\ simulation
program\citep{king10}.  In this chapter, I provide a detailed
description of the simulation procedure, including the theories,
approximations, and assumptions involved.  I also explain the
procedure for extracting kinetic properties of the protein from
experimental data and introduce a quantitative measure of fit quality
between simulation and experimental results.  In addition, the effects
of various experimental parameters on force curve appearance are
demonstrated, and the errors associated with different methods of data
pooling are discussed.  These results should be useful in future
experimental design, artifact identification, and data analysis for
single molecule mechanical unfolding experiments.

\section{Review of current research}

There is a long history of protein unfolding and unbinding
simulations.  Early work by \citet{grubmuller96} and
\citet{izrailev97} focused on molecular dynamics (MD) simulations of
receptor-ligand breakage.  Around the same time, \citet{evans97}
introduced a Monte Carlo Kramers' simulation in the context of
receptor-ligand breakage.  The approach pioneered by \citet{evans97}
was used as the basis for analysis of the initial protein unfolding
experiments\citep{rief97a}.  However, none of these earlier
implementations were open source, which made it difficult to reuse or
validate their results.
%
\nomenclature[text ]{MD}{Molecular dynamics simulation.  Simulate the
  physical motion of atoms and molecules by numerically solving
  Newton's equations.}

Within the Monte Carlo simulation approach, there are two main models
for protein domain unfolding under tension: Bell's and
Kramers'\citep{schlierf06,hummer03,dudko06}.  Bell introduced his
model in the context of cell adhesion\citep{bell78}, but it has been
widely used to model mechanical unfolding in
proteins\citep{rief97a,carrion-vazquez99b,schlierf06} due to its
simplicity and ease of use\citep{hummer03}
(\cref{sec:sawsim:rate:bell}).  Kramers introduced his theory in the
context of thermally activated barrier crossings, which is how we use
it here (\cref{sec:sawsim:rate:other}).

Evans introduced the saddle-point Kramers' approximation in a protein
unfolding context in 1997 (\xref{evans97}{equation}{3}).  However,
early work on mechanical unfolding focused on the simpler Bell
model\citep{rief97a}.  In the early 2000's, the
saddle-point/steepest-descent approximation to Kramer's model
(\xref{hanggi90}{equation}{4.56c}) was introduced into our
field\citep{dudko03,hyeon03}.  By the mid 2000's, the full-blown
double-integral form of Kramer's model
(\xref{hanggi90}{equation}{4.56b}) was in use\citep{schlierf06}.

There have been some tangential attempts towards even fancier models:
\citet{dudko03} attempted to reduce the restrictions of the
single-unfolding-path model and \citet{hyeon03} attempted to measure
the local roughness using temperature dependent unfolding.  However,
further work on these lines has been slow, because the Bell model fits
the data well despite its simplicity.  For more complicated transition
rate models to gain ground, we need larger, more detailed datasets
that expose features which the Bell model doesn't capture.
