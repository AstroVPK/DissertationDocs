\section{Software}
\label{sec:future:software}

Open source experiment control is possible!  Even for a small lab,
with a single novice developer\footnote{
  I started this project with a bit of LabVIEW and Matlab experience,
  but only a few days of Python from the physics department's
  ``Welcome to Drexel'' boot camp.  I stumbled across version control
  on my own, after a year of maintaining a directory full of
  version-stampted tarballs.
},
building reasonable software on top of existing pieces is possible.
After a significant investment in developing
\sawsim\ (\cref{sec:sawsim}), the \pyafm\ stack
(\cref{sec:pyafm,sec:calibcant}), and \Hooke\ (\cref{sec:hooke}), we
have a complete experiment control and analysis suite for single
molecule force spectroscopy.  All of the software in the
stack---including the existing libraries and systems layer
dependencies---is open source, so other labs are free to use, improve,
and republish it as they see fit.
%
\nomenclature[text ]{tarball}{A single file containing a collection of
  files and directories.  Created by
  \href{http://www.gnu.org/software/tar/}{Tar}, tarballs were
  originally used for tape archives (hence the name), but they are now
  often used for distributing project source code.}

As the body of existing science increases, new researchers must become
at the same time more specialized and more interdisciplinary than
their fore-bearers.  With a relatively fixed undergraduate curriculum,
new researchers cannot afford to spend time becoming experts in every
field that bears on their research project.  By pooling resources
between labs, individual researchers can reduce time spent on generic
tooling and increase time spent on their particular project.
Experiment control, analysis, and simulation software is particularly
amenable to community development, because the cost of sharing
software between labs is minimal.

Besides the low cost of transferring the data itself, the rise of
distributed version control systems such as \citetalias{git} have
reduced the administrative overhead of maintaining a project with many
far-flung contributors.  Researchers can automatically fetch and merge
changes made by other groups, incorperating remote improvements.  They
can also commit and push local improvements, which are then available
for remote researchers to incorperate.  The version control systems
and workflows that facilitate this cooperation scale well, from small
projects with a single user, to huge projects like the Linux kernel
with thousands of developers contributing to each release.

Once the software used in a lab has been published, it is also easier
to audit by others who may be skeptical of the summary published in a
journal article.  For example, resolving the confusion about the
``Lorentzian'' (\cref{sec:calibcant:lorentzian}) would be trivial if
\citet{florin95} had also published their explicit procudure for
generating their figure.  Do you think I'm not calibrating my
cantilevers correctly?  Feel free to dig through my code.  Let me know
if you find something wrong (or fix it and send me a patch!).  Science
is built on reproducible experiments and analysis, and open source
software allows you to explicitly specify your methods.  With well
organized code, the specification should be clear from high-level,
experiment-design choices down through low-level bit manipulation.

Many researchers have not received formal training in software
development best practices, so how do we bootstrap this transition to
open source science?  There is a wealth of documentation available
online for self-teaching, and scientists have lots of experience
reading technical writing in their own field.  For those who are
overwhelmed by the amount of available resources, organizations such
as \href{http://http://software-carpentry.org/}{Software Carpentry}
are actively reaching out to scientists with short boot camps to lay
the ground work.  Mastery of any subject takes a significant
investment, but gaining a working level of knowledge in distributed
version control should only take a few days\footnote{
  Software Carpentry allocates half a day to take students from ``What
  is version control?'' to being functioning \citetalias{git} users.
}.  The difficulty for the uninitiated is often not mastering the new
tool or workflow, but learning that it exists at all.  There are a
number of papers highlighting best practices and tools that are good
surveys for guiding future
learning\citep{wilson06a,wilson06b,vandewalle09,aruliah12}.
